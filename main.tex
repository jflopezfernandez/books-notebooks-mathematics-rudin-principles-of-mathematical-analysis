
\documentclass{book}

\usepackage[utf8]{inputenc}
\usepackage[T1]{fontenc}
\usepackage[english]{babel}
\usepackage[paperheight=8.5in,paperwidth=5.5in,outer=0.5in,inner=0.6in,bottom=1in,top=0.7in]{geometry}

\setlength{\parindent}{0em}

\usepackage[backend=biber]{biblatex}
\addbibresource{bibliography.bib}

\usepackage{amsmath}
\usepackage{amsfonts}
\usepackage{amssymb}
\usepackage{amsthm}

\theoremstyle{definition}
\newtheorem*{definition}{Definition}
\newtheorem{theorem}{Theorem}[chapter]
\newtheorem{exercise}{Exercise}[chapter]
\newtheorem*{proposition}{Proposition}

\usepackage{hyperref}

\title{Solutions Notebook for\\Principles of Mathematical Analysis\\by Walter Rudin}
\author{Jose Fernando Lopez Fernandez}
\date{17 March, 2020 -- \today}

\begin{document}
	\frontmatter
	\maketitle
	\tableofcontents
	\mainmatter
	\chapter{The Real and Complex Number Systems}
\setcounter{theorem}{5}
\begin{definition}
	\label{definition-1.6}
	An \textbf{ordered set} is a set $S$ in which an order is defined.
\end{definition}

\begin{definition}
	\label{definition-1.7}
	Suppose $S$ is an ordered set and $E \subset S$. If there exists a $\beta \in S$ such that $x \leq \beta$ for ever $x \in E$, we say that $E$ is \textbf{bounded above} and call $\beta$ an \textbf{upper bound} of $E$.
	\newline\newline
	Lower bounds are defined in the same way, with $\geq$ in place of $\leq$.
\end{definition}

%\setcounter{theorem}{11}
%\begin{definition}
%	\label{definition-1.12}
%	A \textbf{field} is a set $F$ with two operations, called \textit{addition} and \textit{multiplication}, which satisfy the following so-called ''field axioms" (A), (M), and (D):
%	\begin{enumerate}
%		\item 
%	\end{enumerate}
%\end{definition}
\setcounter{theorem}{13}
\begin{pproposition}
	\label{pproposition-1.14}
	The axioms for addition imply the following statements.
	\renewcommand{\labelenumi}{(\alph{enumi})}
	\begin{enumerate}
		\item If $x + y = x + z$ then $y = z$.
		\item If $x + y = x$ then $y = 0$.
		\item If $x + y = 0$ then $y = -x$.
		\item $- \left( -x \right) = x$.
	\end{enumerate}
\end{pproposition}

\begin{pproposition}
	\label{pproposition-1.15}
	The axioms for multiplication imply the following statements.
	\renewcommand{\labelenumi}{(\alph{enumi})}
	\begin{enumerate}
		\item If $x \neq 0$ and $xy = xz$ then $y = z$.
		\item If $x \neq 0$ and $xy = x$ then $y = 1$.
		\item If $x \neq 0$ and $xy = 1$ then $y = \frac{1}{x}$.
		\item If $x \neq 0$ then $\frac{1}{\frac{1}{x}} = x$.
	\end{enumerate}
\end{pproposition}

\section{Exercises}
\begin{exercise}
	If $r$ is rational and $x$ is irrational, prove that $r + x$ and $rx$ are irrational.
\end{exercise}

We will proceed by proving the irrationality of the product $rx$ first, from which the irrationality of the sum $r + x$ will naturally follow.

\begin{proposition}
	For any rational number $r$ and irrational number $x$, the product $rx$ is irrational.
\end{proposition}
\begin{proof}
	Suppose to the contrary that the product $rx$ is rational, implying the existence of some integers $m$ and $n \neq 0$ such that $r = \frac{m}{n}$ and $rx~=~\frac{m}{n}~\cdot~x = \frac{mx}{n}$. If $rx$ was rational, we could multiply by $\frac{1}{r} = \frac{n}{m}$, giving us $\frac{mx}{n} \cdot \frac{n}{m} = x$. Since we have simply multiplied two (allegedly) rational numbers and the rational numbers are a field and thus closed under multiplication, the result must be a rational number. Since the result is $x$, and we began by supposing $x$ was irrational, we've arrived at our contradiction.
	
	The product of a rational number $r$ and an irrational number $x$ yields an irrational number $rx$. 
\end{proof}

\begin{proposition}
	For any rational number $r$ and irrational number $x$, the sum $r + x$ is irrational.
\end{proposition}
\begin{proof}
	Suppose again to the contrary that $r + x$ is rational. Then there exist some integers $m$ and $n$ such that $n \neq 0$ and $r~=~\frac{m}{n}$. By the standard rules of adding rational numbers, this implies the following.
	\begin{align*}
	r + x &= \frac{m}{n} + x \\
	&= \frac{m}{n} + x \cdot \frac{n}{n} \\
	&= \frac{m}{n} + \frac{xn}{n} \\
	&= \frac{m+xn}{n}
	\end{align*}
	Since we've supposed $r + x$ to be a rational number and the field of rational numbers $\mathbb{Q}$ is by definition closed under arithmetic operations, we should be able to add, subtract, multiply, or divide $r + x$ by any non-zero rational number and still get a rational number as a result. To verify this, we begin with the expression $\frac{\left( r + x \right) n - m}{n}$.
	\begin{align*}
	\frac{ \left( r + x \right) n - m }{n} &= \frac{ \left( \frac{ m + nx }{n} \right) n - m }{n} \\
	&= \frac{m + nx - m}{n} \\
	&= \frac{nx}{n} \\
	&= x
	\end{align*}
	Since the above expression simplifies to $x$, an irrational number, the supposition that $r + x$ is rational would imply that the field of rational numbers $\mathbb{Q}$ is not closed under its addition and multiplication operations. This is false, by the definition of a field, and this contradiction leads us to conclude that for any rational number $r$ and irrational number $x$, the sum $r + x$ is irrational.
\end{proof}

\begin{exercise}
	Prove that there is no rational number whose square is~$12$.
\end{exercise}
\begin{proposition}
	There is no rational number whose square is $12$.
\end{proposition}
\begin{proof}
	Let $r$ be some real number such that $r^2 = 12$. Suppose $r$ was rational, such that there existed some non-zero integers $m$ and $n$ such that $r = \frac{m}{n}$ and $\frac{m^2}{n^2} = 12$. Solving for $m$, we get the following.
	\begin{align*}
	\frac{m^2}{n^2} &= 12 \\
	m^2 &= 12n^2 \\
	m^2 &= 3 \cdot 4n^2 \\
	m &= \sqrt{3} \cdot 2n
	\end{align*}
	Since $12$ is even, $12n^2$ must also be even. However, $m$ is clearly not even, as $\sqrt{3} \cdot 2n$ is not evenly divisible by $2$. This means that $m^2$ must be odd, in direct contradiction with our earlier assertion that $12n^2$ must be even. We conclude from this contradiction that there cannot exist a rational number whose square is $12$.
\end{proof}

\begin{exercise}
	Prove Proposition~\ref{pproposition-1.15}.
\end{exercise}
Let $x$, $y$, and $z$ be arbitrary elements of some field $F$.
\begin{proposition}
	If $x \neq 0$ and $xy = xz$ then $y = z$.
\end{proposition}
\begin{proof}
	Suppose $x \neq 0$ and $xy = xz$. By M5, there exists some element $x^{-1} \in F$ such that $x \cdot x^{-1} = 1$. Since multiplication in a field is both commutative and associative, the order of the operation or the factors is immaterial.
	\begin{align*}
	xy &= xz &\\
	x^{-1}\left(xy\right) &= x^{-1}\left(xz\right) &\text{By M5}\\
	\left( x^{-1}x \right)y &= \left( x^{-1}x \right) z &\text{By M3}\\
	1y &= 1z &
	\end{align*}
	Finally, M4 asserts the existence of an identity element $1 \in F$ such that $1 \cdot x = x$ for any element $x$ in $F$. Therefore,
	\begin{align*}
	1y &= y, \text{ and} \\
	1z &= z.
	\end{align*}
	Therefore,
	\begin{equation*}
	y = z.
	\end{equation*}
\end{proof}

\begin{proposition}
	If $x \neq 0$ and $xy = x$ then $y = 1$.
\end{proposition}
\begin{proof}
	Suppose $x \neq 0$ and $xy = x$. By M5, there exists some non-zero element $x^{-1} \in F$ such that $x^{-1}x = 1$. Since multiplication in a field is both associative and commutative, the order of the factors is immaterial to the result. Simplifying, we conclude that since $1y = y$ by M4, $y$ must equal $1$.
	\begin{align*}
	xy &= x & \\
	x^{-1}xy &= x^{-1}x &\text{By M5}\\
	1y &= 1 &\text{By M4}\\
	y &= 1 &
	\end{align*}
\end{proof}

\begin{proposition}
	If $x \neq 0$ and $xy = 1$ then $y = \frac{1}{x}$.
\end{proposition}
\begin{proof}
	Suppose $x \neq 0$ and $xy = 1$. The fifth axiom of multiplication in a field (M5) asserts the existence of a multiplicative inverse for every non-zero element in the field. Since we have assumed $x \neq 0$, we multiply both sides of the equation to obtain the desired result. Once again, M2 and M3 ensure the order of operations and factors is irrelevant.
	\begin{align*}
	xy &= 1 & \\
	x^{-1}xy &= x^{-1}1 &\text{By M5} \\
	1y &= 1x^{-1} &\\
	y &= x^{-1} &\text{By M4}
	\end{align*}
\end{proof}
The above result makes sense, as a field is formally defined as a commutative ring with unity with no zero-divisors and in which every non-zero element is also a unit. It should also be noted that if $y = \frac{1}{x}$ and $y = x$ are both true, then $x = y = 1$.

\begin{proposition}
	If $x \neq 0$ then $\frac{1}{\frac{1}{x}} = x$.
\end{proposition}
\begin{proof}
	Suppose $x \neq 0$ and $y$ is some element in $F$ such that $y = \frac{1}{x}$. By M5, $xy = 1$. Therefore,
	\begin{align*}
	xy &= 1 & \\
	xyy^{-1} &= 1 \cdot y^{-1} &\text{By M2, M3, and M5} \\
	1x &= \frac{1}{y} &\text{By M5}\\
	x &= \frac{1}{y} &\text{By M4}\\
	x &= \frac{1}{\frac{1}{x}}. &
	\end{align*}
\end{proof}

\begin{exercise}
	Let $E$ be a nonempty subset of an ordered set; suppose $\alpha$ is a lower bound of $E$ and $\beta$ is an upper bound of $E$. Prove that $\alpha \leq \beta$.
\end{exercise}
\begin{proposition}
	Let $E$ be a non-empty subset of some ordered set $S$, and suppose that $\alpha$ and $\beta$ are lower and upper bounds of $E$, respectively. Then $\alpha \leq \beta$.
\end{proposition}
\begin{proof}
	The existence of some upper bound $\beta$ implies that $E$ is bounded above, and therefore there exists some element(s) $\beta_0 \in S$ such that $\beta_0 \geq x$ for all $x \in E$. Let $U$ denote the set of all such elements. Clearly, $\beta \in U$. Let $\beta_\text{min}$ be some element in $U$ such that $\beta_\text{min} \leq \beta_0$ for all $\beta_0 \in U$. By Definition~1.8, $\beta_\text{min} = \sup E$.
	\newline\newline
	Analogously, the existence of some lower bound $\alpha$ implies that $E$ is bounded below, and therefore there exists some element(s) $\alpha_0 \in S$ such that $\alpha_0 \leq x$ for all $x \in E$. Let $L$ denote the set of all such elements. Again, $\alpha \in L$. Let $\alpha_\text{max}$ be some element in $L$ such that $\alpha_\text{max} \geq \alpha_0$ for all $\alpha_0 \in L$. By Definition~1.8, $\alpha_\text{max} = \inf E$.
	\newline\newline
	Since $\alpha$ and $\beta$ were not specifically designated to be the greatest lower bound or least upper bound of $E$, respectively, we must treat them with the requisite generality. By the definitions above, $\alpha_0 \leq \alpha_\text{max}$ for all $\alpha_0 \in L$ and $\beta_\text{min} \leq \beta_0$ for all $\beta_0 in U$, and therefore $\alpha \leq \alpha_\text{max}$ and $\beta_\text{min} \leq \beta_0$ for any arbitrary lower and upper bounds of $E$, respectively.
	\newline\newline
	By the definition of the infimum and supremum of an ordered set, $\inf E \leq x \leq \sup E$ for any $x \in E$, implying that
	\begin{equation}
	\label{equation-chapter-1-exercise-4-1}
	\alpha \leq \inf E \leq x \leq \sup E \leq \beta,
	\end{equation}
	and therefore $\alpha \leq \beta$, as required.
\end{proof}
\begin{remark}
	It should be noted that in~\eqref{equation-chapter-1-exercise-4-1}, equality holds only in the degenerate case were the subset $E$ of $S$ consists of a single element, and even then only if $\alpha = \inf E$ and $\beta = \sup E$, which need not be the case.
\end{remark}

\begin{exercise}
	Let $A$ be a nonempty set of real numbers which is bounded below. Let $-A$ be the set of all numbers $-x$, where $x \in A$. Prove that
	\begin{equation*}
	\inf A = - \sup \left( -A \right).
	\end{equation*}
\end{exercise}
\begin{proposition}
	Let $A$ be a nonempty set of real numbers which is bounded below, and let $-A$ be the set of all numbers $-x$, where $x \in A$. Then, $\inf A = - \sup \left( -A \right)$.
\end{proposition}
\begin{proof}
	content...
\end{proof}

\begin{exercise}
	Fix $b > 1$.
	\renewcommand{\labelenumi}{(\alph{enumi})}
	\begin{enumerate}
		\item If $m,n,p,q$ are integers, $n > 0$, $q > 0$, and $r~=~\frac{m}{n}~=~\frac{p}{q}$, prove that
		\begin{equation*}
		\left( b^m \right)^{\frac{1}{n}} = \left( b^p \right)^{\frac{1}{q}}.
		\end{equation*}
		\item Prove that $b^{r + s} = b^rb^s$ if $r$ and $s$ are rational.
		\item If $x$ is real, define $B \left( x \right)$ to be the set of all numbers $b^t$, where $t$ is rational and $t \leq x$. Prove that
		\begin{equation*}
		b^r = \sup B \left( r \right)
		\end{equation*}
		when $r$ is rational. Hence it makes sense to define
		\begin{equation*}
		b^x = \sup B \left( x \right)
		\end{equation*}
		for every real $x$.
		\item Prove that $b^{r + x} = b^xb^y$ for all reall $x$ and $y$.
	\end{enumerate}
\end{exercise}

\begin{exercise}
	Fix $b > 1$, $y > 0$, and prove that there is a unique real $x$ such that $b^x = y$, by completing the following outline. (This $x$ is called the \textit{logarithm of $y$ to the base $b$}.)
	\renewcommand{\labelenumi}{(\alph{enumi})}
	\begin{enumerate}
		\item For any positive integer $n$, $b^n - 1 \geq n \left( b - 1 \right)$.
		\item Hence $b - 1 \geq n \left( b^{\frac{1}{n}} - 1 \right)$.
		\item If $t > 1$ and $n > \frac{\left( b - 1 \right)}{\left( t - 1 \right)}$, then $b^{\frac{1}{n}} < t$.
		\item If $w$ is such that $b^w < y$, then $b^{w + \frac{1}{n}} < y$ for sufficiently large $n$; to see this, apply part (c) with $t = y \cdot b^{-w}$.
		\item If $b^w > y$, then $b^{w - \frac{1}{n}} > y$ for sufficiently large $n$.
		\item Let $A$ be the set of all $w$ such that $b^w < y$, and show that $x = \sup A$ satisfies $b^x = y$.
		\item Prove that this $x$ is unique.
	\end{enumerate}
\end{exercise}

\begin{exercise}
	Prove that no order can be defined in the complex field that turns it into an ordered field. \textit{Hint:} $-1$ is a square.
\end{exercise}

\begin{exercise}
	Suppose $z = a + bi$, $w = c + di$. Define $z < w$ if $a < c$ and also if $a = c$ but $b < d$. Prove that this turns the set of all complex numbers into an ordered set. (This type of order relation is called a \textit{dictionary order}, or \textit{lexicographic order}, for obvious reasons.) Does this ordered set have the least-upper-bound property?
\end{exercise}

\setcounter{exercise}{17}
\begin{exercise}
	If $k \geq 2$ and $\textbf{x} \in \mathbb{R}^k$, prove that there exists $\textbf{y} \in \mathbb{R}^k$ such that $\textbf{y} \neq \textbf{0}$ but $\textbf{x} \cdot \textbf{y} = 0$. Is this also true if $k = 1$?
\end{exercise}
\begin{proposition}
	Let $k \geq 2$ and $\textbf{x} \in \mathbb{R}^k$. There exists some $\textbf{y} \in \mathbb{R}^k$ where $\textbf{y} \neq \textbf{0}$ such that $\textbf{x} \cdot \textbf{y} = 0$.
\end{proposition}
\begin{proof}
	Let $\textbf{x} = \left( 1, 0 \right)$ and $\textbf{y} = \left( 0, 1 \right)$. By the definition of the inner product of two vectors,
	\begin{align*}
	\textbf{x} \cdot \textbf{y} &= \sum_{i=1}^{2} x_iy_i \\
	&= x_1y_1 + x_2y_2 \\
	&= \left( 1 \cdot 0 \right) + \left( 0 \cdot 1 \right) \\
	&= \left( 0 \right) + \left( 0 \right) \\
	&= 0.
	\end{align*}
	Therefore, there exists some $\textbf{y} \in \mathbb{R}^k$ where $\textbf{y} \neq \textbf{0}$ such that $\textbf{x} \cdot \textbf{y} = 0$.
\end{proof}

This is not true for vector spaces of dimension one.


	\chapter{Basic Topology}

\setcounter{theorem}{11}
\begin{theorem}
	\label{theorem-2.12}
	Let $\left\lbrace E_n \right\rbrace$ for $n = 1, 2, 3, \ldots$ be a sequence of countable sets, and put
	\begin{equation}
	S = \bigcup_{n=1}^\infty E_n.
	\end{equation}
	Then $S$ is countable.
\end{theorem}

\begin{corollary}
	\label{theorem-2.12-corollary-1}
	Suppose $A$ is at most countable, and, for every $\alpha \in A$, $B_\alpha$ is at most countable. Put
	\begin{equation*}
	T = \bigcup_{\alpha \in A} B_\alpha.
	\end{equation*}
	Then $T$ is at most countable.
\end{corollary}

\begin{theorem}
	\label{theorem-2.13}
	\index{countable}
	Let $A$ be a countable set, and let $B_n$ be the set of all $n$-tuples $\left( a_1, \ldots, a_n \right)$, where $a_k \in A$ for $\left( k = 1, \ldots, n \right)$, and the elements $a_1, \ldots, a_n$ need not be distinct. Then $B_n$ is countable.
\end{theorem}

\setcounter{theorem}{18}
\begin{theorem}
	\label{theorem-2.19}
	Every neighborhood is an open set.
\end{theorem}

\begin{theorem}
	\label{theorem-2.20}
	If $p$ is a limit point of a set $E$, then every neighborhood of $p$ contains infinitely many points of $E$.
\end{theorem}

\setcounter{theorem}{22}
\begin{theorem}
	\label{theorem-2.23}
	A set $E$ is open if and only if its complement is closed.
\end{theorem}

\begin{corollary}
	\label{theorem-2.23-corollary-1}
	A set $F$ is closed if and only if its complement is open.
\end{corollary}

\setcounter{theorem}{30}
\begin{definition}
	\label{theorem-2.31}
	By an \textbf{\gls{open cover}} of a set $E$ in a metric space $X$ we mean a collection $\left\lbrace G_\alpha \right\rbrace$ of open subsets of $X$ such that $E \subset \cup_\alpha G_\alpha$.
\end{definition}

\begin{definition}
	A subset $K$ of a metric space $X$ is said to be \textbf{\gls{compact}} if every open cover of $K$ contains a \textit{finite} subcover.
\end{definition}

\setcounter{theorem}{40}
\begin{theorem}
	\label{theorem-2.41}
	If a set in $\mathbb{R}^k$ has one of the following three properties, then it has the other two:
	\renewcommand{\labelenumi}{(\alph{enumi})}
	\begin{enumerate}
		\item $E$ is closed and bounded.
		\item $E$ is compact.
		\item Every infinite subset of $E$ has a limit point in $E$.
	\end{enumerate}
\end{theorem}

\section{Exercises}
\begin{exercise}
	Prove that the empty set is a subset of every set.
\end{exercise}
\begin{proposition}
	The empty set, denoted by $\emptyset$, is a subset of every set.
\end{proposition}
\begin{proof}
	Let $A$ be an arbitrary set. In order to prove that the empty set is a subset of $A$, we must show that for any element $x$ in $\emptyset$, $x$ is also an element of $A$. Since --by definition-- the empty set contains no elements, this statement is vacuously true and thus not very interesting.
	\newline\newline
	Consider instead the contrapositive: for any element $x$, if $x$ is not an element of $A$, then $x$ is not an element of $\emptyset$. Again, since $\emptyset$ contains no elements, this statement is true for all elements $x \notin A$, as required.
\end{proof}

\begin{exercise}
	A complex number $z$ is said to be \textit{algebraic} if there are integers $a_0,\ldots,a_n$, not all zero, such that
	\begin{equation*}
	a_0z^n + a_1z^{n-1} + \cdots + a_{n-1}z + a_n = 0.
	\end{equation*}
	Prove that the set of all algebraic numbers is countable. \textit{Hint:} For every positive integer $N$ there are only finitely many equations with
	\begin{equation*}
	n + |a_0| + |a_1| + \cdots + |a_n| = N.
	\end{equation*}
\end{exercise}
\begin{proposition}
	The set of all algebraic numbers is countable.
\end{proposition}
%\begin{proof}
%	content...
%\end{proof}

\begin{exercise}
	Prove that there exist real numbers which are not algebraic.
\end{exercise}

\begin{exercise}
	Is the set of all irrational real numbers countable?
\end{exercise}


\setcounter{exercise}{11}
\begin{exercise}
	Let $K \subset \mathbb{R}^1$ consist of $0$ and the numbers $\frac{1}{n}$, for $n = 1, 2, 3, \ldots$. Prove that $K$ is compact directly from the definition (without using the Heine-Borel Theorem).
\end{exercise}
\begin{proposition}
	Let $K \subset \mathbb{R}^1$ consist of $0$ and the numbers $\frac{1}{n}$ for $n~=~1, 2, 3,~\ldots$. The space $K$ is compact.
\end{proposition}
%\begin{proof}
%	Let $I_n$ be an arbitrary covering set of $K$ composed of the neighborhoods of every point $k \in K$ of radius $\frac{1}{n}$ for any $n \in \mathbb{N}$. Since the distance between two consecutive points $\frac{1}{n}$ and $\frac{1}{n+1}$ is defined by the metric $\left| \frac{1}{n+1} - \frac{1}{n} \right| = \left| \frac{1}{n(n+1)} \right|$, and $\frac{1}{n(n+1)} \leq \frac{1}{n}$ for all $n \in \mathbb{N}$, every open cover $I_n$ of $K$ has a finite subcover, and therefore $K$ is compact.
%\end{proof}
\begin{proof}
	Any open cover must include the interval $a < 0 < b$. By the Archimedean property of $\mathbb{R}$, there are an infinite number of points $k \in K$ between $0$ and $b$, irrespective of the value of $b$. The covering set for the rest of the points in $K$ from $b$ to $1$ is therefore finite.
	\newline\newline
	Since any cover of $K$ contains some finite subcover, $K$ is compact.
\end{proof}

\setcounter{exercise}{15}
\begin{exercise}
	Regard $\mathbb{Q}$, the set of all rational numbers, as a metric space, with $d(p,q) = |p - q|$. Let $E$ be the set of all $p \in \mathbb{Q}$ such that $2 < p^2 < 3$. Show that $E$ is closed and bounded in $\mathbb{Q}$, but that $E$ is not compact. Is $E$ open in $\mathbb{Q}$?
\end{exercise}


	\chapter{Numerical Sequences and Series}

\section{Exercises}
\setcounter{exercise}{24}
\begin{exercise}
	Let $X$ be the metric space whose points are the rational numbers, with the metric $d(x,y) = |x-y|$. What is the completion of this space? (Compare Exercise 24.)
\end{exercise}
\begin{solution}
	The completion of this space is the real numbers, within which the rational numbers are dense. In fact, one of the axiomatic constructions of the real numbers is precisely the completion of the rational numbers by the use of Cauchy sequences.
\end{solution}


	\backmatter
	\printbibliography[heading=bibintoc]
\end{document}
