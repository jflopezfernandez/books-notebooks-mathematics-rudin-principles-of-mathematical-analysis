\chapter{Basic Topology}

\setcounter{theorem}{11}
\begin{theorem}
	\label{theorem-2.12}
	Let $\left\lbrace E_n \right\rbrace$ for $n = 1, 2, 3, \ldots$ be a sequence of countable sets, and put
	\begin{equation}
	S = \bigcup_{n=1}^\infty E_n.
	\end{equation}
	Then $S$ is countable.
\end{theorem}

\begin{corollary}
	\label{theorem-2.12-corollary-1}
	Suppose $A$ is at most countable, and, for every $\alpha \in A$, $B_\alpha$ is at most countable. Put
	\begin{equation*}
	T = \bigcup_{\alpha \in A} B_\alpha.
	\end{equation*}
	Then $T$ is at most countable.
\end{corollary}

\begin{theorem}
	\label{theorem-2.13}
	\index{countable}
	Let $A$ be a countable set, and let $B_n$ be the set of all $n$-tuples $\left( a_1, \ldots, a_n \right)$, where $a_k \in A$ for $\left( k = 1, \ldots, n \right)$, and the elements $a_1, \ldots, a_n$ need not be distinct. Then $B_n$ is countable.
\end{theorem}

\setcounter{theorem}{18}
\begin{theorem}
	\label{theorem-2.19}
	Every neighborhood is an open set.
\end{theorem}

\begin{theorem}
	\label{theorem-2.20}
	If $p$ is a limit point of a set $E$, then every neighborhood of $p$ contains infinitely many points of $E$.
\end{theorem}

\setcounter{theorem}{22}
\begin{theorem}
	\label{theorem-2.23}
	A set $E$ is open if and only if its complement is closed.
\end{theorem}

\begin{corollary}
	\label{theorem-2.23-corollary-1}
	A set $F$ is closed if and only if its complement is open.
\end{corollary}

\setcounter{theorem}{27}
\begin{theorem}
	\label{theorem-2.28}
	Let $E$ be a nonempty set of real numbers which is bounded above. Let $y = \text{sup} E$. Then $y \in \overline{E}$. Hence $y \in E$ if $E$ is closed.
\end{theorem}

\setcounter{theorem}{30}
\begin{definition}
	\label{theorem-2.31}
	By an \textbf{\gls{open cover}} of a set $E$ in a metric space $X$ we mean a collection $\left\lbrace G_\alpha \right\rbrace$ of open subsets of $X$ such that $E \subset \cup_\alpha G_\alpha$.
\end{definition}

\begin{definition}
	A subset $K$ of a metric space $X$ is said to be \textbf{\gls{compact}} if every open cover of $K$ contains a \textit{finite} subcover.
\end{definition}

\setcounter{theorem}{34}
\begin{theorem}
	\label{theorem-2.35}
	Closed subsets of compact sets are compact.
\end{theorem}

\setcounter{theorem}{40}
\begin{theorem}
	\label{theorem-2.41}
	If a set in $\mathbb{R}^k$ has one of the following three properties, then it has the other two:
	\renewcommand{\labelenumi}{(\alph{enumi})}
	\begin{enumerate}
		\item $E$ is closed and bounded.
		\item $E$ is compact.
		\item Every infinite subset of $E$ has a limit point in $E$.
	\end{enumerate}
\end{theorem}

\setcounter{theorem}{46}
\begin{theorem}
	\label{theorem-2.47}
	A subset $E$ of the real line $\mathbb{R}^1$ is connected if and only if it has the following property: if $x \in E$, $y \in E$, and $x < z < y$, then $z \in E$.
\end{theorem}

\section{Exercises}
\begin{exercise}
	Prove that the empty set is a subset of every set.
\end{exercise}
\begin{proposition}
	The empty set, denoted by $\emptyset$, is a subset of every set.
\end{proposition}
\begin{proof}
	Let $A$ be an arbitrary set. In order to prove that the empty set is a subset of $A$, we must show that for any element $x$ in $\emptyset$, $x$ is also an element of $A$. Since --by definition-- the empty set contains no elements, this statement is vacuously true and thus not very interesting.
	\newline\newline
	Consider instead the contrapositive: for any element $x$, if $x$ is not an element of $A$, then $x$ is not an element of $\emptyset$. Again, since $\emptyset$ contains no elements, this statement is true for all elements $x \notin A$, as required.
\end{proof}

\begin{exercise}
	A complex number $z$ is said to be \textit{algebraic} if there are integers $a_0,\ldots,a_n$, not all zero, such that
	\begin{equation*}
	a_0z^n + a_1z^{n-1} + \cdots + a_{n-1}z + a_n = 0.
	\end{equation*}
	Prove that the set of all algebraic numbers is countable. \textit{Hint:} For every positive integer $N$ there are only finitely many equations with
	\begin{equation*}
	n + |a_0| + |a_1| + \cdots + |a_n| = N.
	\end{equation*}
\end{exercise}
\begin{proposition}
	The set of all algebraic numbers is countable.
\end{proposition}
%\begin{proof}
%	content...
%\end{proof}

\begin{exercise}
	Prove that there exist real numbers which are not algebraic.
\end{exercise}

\begin{exercise}
	Is the set of all irrational real numbers countable?
\end{exercise}


\setcounter{exercise}{11}
\begin{exercise}
	Let $K \subset \mathbb{R}^1$ consist of $0$ and the numbers $\frac{1}{n}$, for $n = 1, 2, 3, \ldots$. Prove that $K$ is compact directly from the definition (without using the Heine-Borel Theorem).
\end{exercise}
\begin{proposition}
	Let $K \subset \mathbb{R}^1$ consist of $0$ and the numbers $\frac{1}{n}$ for $n~=~1, 2, 3,~\ldots$. The space $K$ is compact.
\end{proposition}
%\begin{proof}
%	Let $I_n$ be an arbitrary covering set of $K$ composed of the neighborhoods of every point $k \in K$ of radius $\frac{1}{n}$ for any $n \in \mathbb{N}$. Since the distance between two consecutive points $\frac{1}{n}$ and $\frac{1}{n+1}$ is defined by the metric $\left| \frac{1}{n+1} - \frac{1}{n} \right| = \left| \frac{1}{n(n+1)} \right|$, and $\frac{1}{n(n+1)} \leq \frac{1}{n}$ for all $n \in \mathbb{N}$, every open cover $I_n$ of $K$ has a finite subcover, and therefore $K$ is compact.
%\end{proof}
\begin{proof}
	Any open cover must include the interval $a < 0 < b$. By the Archimedean property of $\mathbb{R}$, there are an infinite number of points $k \in K$ between $0$ and $b$, irrespective of the value of $b$. The covering set for the rest of the points in $K$ from $b$ to $1$ is therefore finite.
	\newline\newline
	Since any cover of $K$ contains some finite subcover, $K$ is compact.
\end{proof}

\setcounter{exercise}{15}
\begin{exercise}
	Regard $\mathbb{Q}$, the set of all rational numbers, as a metric space, with $d(p,q) = |p - q|$. Let $E$ be the set of all $p \in \mathbb{Q}$ such that $2 < p^2 < 3$. Show that $E$ is closed and bounded in $\mathbb{Q}$, but that $E$ is not compact. Is $E$ open in $\mathbb{Q}$?
\end{exercise}

