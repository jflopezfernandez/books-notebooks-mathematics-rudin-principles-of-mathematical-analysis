\begin{exercise}
	Suppose $f$ is a real function defined on $\mathbb{R}^1$ which satisfies
	\begin{equation*}
	\lim\limits_{h \to 0} \left[ f\left(x+h\right) - f\left(x-h\right) \right] = 0
	\end{equation*}
	for every $x \in \mathbb{R}^1$. Does this imply that $f$ is continuous?
\end{exercise}
\begin{solution}
	No. The statement above is akin to saying
	\begin{equation*}
	\lim\limits_{h \to 0} f\left(x\right) - \lim\limits_{h \to 0} f\left(x\right) = 0,
	\end{equation*}
	which is not much of a statement. Not only have we not proven anything about the continuity of $f$, we haven't proven anything at all, really.
\end{solution}
